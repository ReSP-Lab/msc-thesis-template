%% This template is based on the work of Ruben De Smet (https://gitlab.com/rubdos/texlive-vub) and Nick Van Goethem. It is altered further to create a more general template for the Master thesis's report submission within the Master in CyberSecurity program. Modified files (such as brufaceStyle.sty and brufaceMScThesis.tex have been renamed. The others have been left as is on purpose for easy update.)
%% Further alterations implemented by Jérôme Dossogne
%% License: LaTeX Project Public License 1.3c

%% According to the Master thesis regulations, the students must hand in a double-sided hard copy version of the Master thesis to each member of the jury + ONE for the secretariat, while also forwarding an electronic version to the faculty Secretariat
%% The students shall also send a complete electronic archive (source, references, code, documentation) to the project owner before the defense
%% To save the according PDF's you must type in the following document-classes for the:
%	- electronic version
%\documentclass[11pt,notitlepage]{report}
%	- double-sided (to print) version

\documentclass[11pt,twoside,notitlepage]{report}%report, book, memoir https://ctan.org/pkg/memoir?lang=en
%% and save/download each run seperately
\usepackage{styles/cyberSecurityStyle}
%% you can add other languages by changing or adding them to the babel-package between []
\usepackage[T1]{fontenc}
\usepackage[english,french,dutch]{babel}
\usepackage{helvet} % helvetica font for front page and abstract heading
\usepackage{amsmath,amssymb} % provides the symbols for mathematical expressions
\usepackage{siunitx} % provides the SI units
\usepackage{titlesec} % to adjust the spacing before and after the chapter
\usepackage{graphicx}
\usepackage{wrapfig}
\usepackage{cite} % regroup multiple citations
\usepackage{epsfig}
\usepackage{tabularx}
\usepackage{multirow}
\usepackage{tablefootnote}
\usepackage{subcaption}
\usepackage{setspace}
\usepackage{sectsty}
\usepackage[titles]{tocloft}
\setlength{\cftbeforechapskip}{5pt} % reduces the spacing between chapters in the table of content
\usepackage{lipsum}
\usepackage[hidelinks,pagebackref=true]{hyperref} % NEED FIX: backref does not display when a citation is on multiple page
\renewcommand*{\backref}[1]{}
\renewcommand*{\backrefalt}[4]{[{\tiny%
    \ifcase #1 Not cited.%
          \or Cited on page~#2.%
          \else Cited on pages #2.%
    \fi%
    }]}
\usepackage{url}% for url's in bib
\usepackage{cleveref}
\usepackage[acronyms,toc,nonumberlist,nogroupskip,automake,nopostdot]{glossaries}
\usepackage{listings}
\usepackage{fancyhdr}
\usepackage{tikz}
\usepackage{soul}
\def\textBF#1{\sbox\CBox{#1}\resizebox{\wd\CBox}{\ht\CBox}{\textbf{#1}}}
\usepackage[inline]{enumitem}
\usepackage{pdfpages}
\usepackage[tikz]{bclogo}
\definecolor{bgblue}{RGB}{245,243,253}
\definecolor{ttblue}{RGB}{1,15,142}
\definecolor{grey}{RGB}{238,238,238}
\definecolor{orange}{RGB}{139,60,0}

\sisetup{detect-all}



\pagestyle{fancy}
\fancyhf{}
\setlength{\headheight}{14pt}
\fancyhead[RE]{\rightmark}
\fancyhead[LO]{\leftmark}
\fancyhead[LE,RO]{\thepage}
\fancyfoot[RO]{\begin{picture}(0,0) \put(20,755){\vubtriangle} \end{picture}}
\fancyfoot[LE]{\begin{picture}(0,0) \put(-33,766){\ulbtriangle} \end{picture}}

\makeatletter
%%% redefine \eqref to be like the original
\renewcommand{\eqref}[1]{\textup{\eqreftagform@{\ref{#1}}}}
\let\eqreftagform@\tagform@
%%% redefine \tagform@
\def\tagform@#1{%
	\maketag@@@{%
		\if@unit\ensuremath{\left[\thiseq@unit\right]}\quad\fi\global\@unitfalse
		(\ignorespaces#1\unskip\@@italiccorr)%
	}%
}
\newif\if@unit
\def\equnit#1{%
	\gdef\thiseq@unit{#1}%
	\global\@unittrue
}
\makeatother

%% adjusts the spacing above and below the chapter title
\titleformat{\chapter}[display]{\normalfont\LARGE\bfseries}{\chaptertitlename\ \thechapter}{0pt}{\Huge}
\titlespacing*{\chapter}{0pt}{0pt}{10pt}

%% redefines the title of \right) he table of contents
\addto\captionsenglish{% Replace "english" with the language you use
	\renewcommand{\contentsname}%
	{Table of Contents}%
}
%% ---------------------------------------Title Page---------------------------------------
\title{<Insert Main Title Here>}
\subtitle{<Insert Sub Title Here>}
\author{Author1 \newline Author2}
%% general information about your promotor(s) and academic year (TO BE COMPLETED)
\academicYear{Year1 -- Year1}
\promotor{<Promotor>}
\projectOwner{<Project Owner>}
% \coSupervisor{Prof.\ Dr.\ Ir.\ Knows Less}
%% uncomment the Master program (and its major) to display in the document
%\masterAndMajor{Master of Science in Architectural Engineering}{}
%\masterAndMajor{Master of Science in Chemical and Materials Engineering}{major in Materials}
%\masterAndMajor{Master of Science in Chemical and Materials Engineering}{major in Process Technology}
%\masterAndMajor{Master of Science in Civil Engineering}{}
%\masterAndMajor{Master of Science in Electromechanical Engineering}{major in Aeronautics}
%\masterAndMajor{Master of Science in Electromechanical Engineering}{major in Energy}
%\masterAndMajor{Master of Science in Electromechanical Engineering}{major in Mechatronics-Construction}
\masterAndMajor{Master in Cybersecurity}{<Corporate Strategies/System Design>}
%\masterAndMajor{Master of Science in Electrical Engineering}{}

%% There are two different defined geometries, one for the electronic version, and one for the double-sided print-version (CHOOSE that one you would like to save)
%	- electronic version
\geometry{top=2.5cm,bottom=2.25cm,left=3cm,right=3cm}
%	- double-sided print version
%\geometry{top=2.5cm,bottom=2.5cm,inner=3.5cm,outer=2.5cm}

%\allsectionsfont{\normalfont\sffamily\bfseries} % applies sans serif font to the headings
% ----------------- here we make the list of symbols and abbreviations --------------------
\newglossary[slg]{symbols}{sym}{sbl}{List of Symbols} % create add. symbolslist
%% here we define the symbol listing environment, DO NOT ALTER !!!
\glsaddkey{unit}{\glsentrytext{\glslabel}}{\glsentryunit}{\GLsentryunit}{\glsunit}{\Glsunit}{\GLSunit}\glssetnoexpandfield{unit}
\setlength{\glsdescwidth}{15cm} % sets the width of the glossary environment
\newglossarystyle{symbunitlong}{%
	\setglossarystyle{super3col}% base this style on the list style
	\renewenvironment{theglossary}{% Change the table type --> 3 columns
		\begin{supertabular}{p{8mm}p{0.8\glsdescwidth}>{\centering\arraybackslash}p{2cm}}}%
		{\end{supertabular}}%
	%
%	\renewcommand*{\glossaryheader}{%  Change the table header
%		\bfseries Symbol & \bfseries Description & \bfseries Unit \\
%		\endhead}
	\renewcommand*{\glossentry}[2]{%  Change the displayed items
		\glstarget{##1}{\glossentryname{##1}} %
		& \glossentrydesc{##1}% Description
		& \glsunit{##1}  \tabularnewline
	}
}
%%-------Here starts the file in which the actual acronyms and symbols are putted in--------
%--------------------------------------------------------------------------
%% Below, we define all the acronyms we want to use throughout the document
%--------------------------------------------------------------------------
\newacronym{egr}{EGR}{Exhaust Gas Recirculation}
\newacronym{doc}{DOC}{Diesel Oxidation Catalyst}
\newacronym{hev}{HEV}{Hybrid Electric Vehicle}
%--------------------------------------------------------------------------
%% Below, we define all the symbols we want to use throughout the document
%--------------------------------------------------------------------------
\newglossaryentry{symb:Pi}{name=\ensuremath{\pi},
	description={Geometrical value},
	unit={-},
	type=symbols}
%
\newglossaryentry{height}{name=\ensuremath{h},
	description={Height of the tower},
	unit={\si{\metre}},
	type=symbols}
%
\newglossaryentry{energyconsump}{name=\ensuremath{P},
	description={Power consumption (and another very long definition)},
	unit={\si{\kilo\watt}},
	type=symbols}
%
\newglossaryentry{density}{name=\ensuremath{\rho},
	description={Density of a fluid},
	unit={\si[per-mode=symbol-or-fraction]{\kilogram\per\cubic\metre}},
	type=symbols}
%
\newglossaryentry{width}{name=\ensuremath{w},
	description={Width of the tower},
	unit={\si{\metre}},
	type=symbols}
%
\newglossaryentry{energyproduction}{name=\ensuremath{E},
	description={Energy production (and another very long definition)},
	unit={\si{\kilo\joule}},
	type=symbols}
\makeglossaries

%% -------------------Here starts all the basic code for the document-----------------------
\begin{document}

\maketitle

\pagestyle{empty} 

\textcolor{white}{This is done in order to skip the first half of the page}\\
\fontsize{10}{12}

%\vspace{13cm}
\vspace{15cm}% when you would not have a confidentiality claus

%% if you have a confidentiaity claus with the university or a third party, uncomment the text below and fill in the proper embargo date
%\noindent\textsf{Confidential up to and including dd/mm/20xx}\\
%\noindent\textsf{Important}\\
%
%\vspace{5mm}
%
%\noindent\textsf{This master dissertation contains confidential information and/or confidential research results proprietary to the \textit{Vrije Universiteit Brussel}, \textit{Universit\'{e} Libre de Bruxelles} or third parties. It is strictly forbidden to publish, cite or make public in any way this master’s dissertation or any part thereof without the written permission of the author(s), the \textit{Vrije Universiteit Brussel}, the \textit{Universit\'{e} Libre de Bruxelles} and/or the third parties involved.}\\ 
%
%\noindent\textsf{Under no circumstance may this master dissertation be communicated to or put at the disposal of other third parties. Photocopying or duplicating it in any other way is strictly prohibited.
%Disregarding the confidential nature of this master dissertation may cause irremediable damage to the \textit{Vrije Universiteit Brussel}, \textit{Universit\'{e} Libre de Bruxelles} or third parties involved.}\\
%\noindent\textsf{The stipulations mentioned above are in force until the embargo date.}\\
%
%\vspace{5mm}

\noindent\textsf{I hereby confirm that this thesis was written independently by myself without the use of any sources beyond those cited, and all passages and ideas taken from other sources are cited accordingly.}\\

\noindent\textsf{The author(s) gives (give) permission to make this master dissertation available for consultation and to copy parts of this master dissertation for personal use. In all cases of other use, the copyright terms have to be respected, in particular with regard to the obligation to state explicitly the source when quoting results from this master dissertation.}\\

\noindent\textsf{The author(s) transfers (transfer) to the project owner(s) any and all rights to this master dissertation, code and all contribution to the project without any limitation in time nor space.}\\

%if no particular confidentianlity has been used, just use the date of the thesis' deadline
\noindent\textsf{dd/mm/20xx}

\clearpage
%% front matter
\pagenumbering{Roman} % Roman page numbering for the front matter
\pagestyle{plain}
%% here, the files for the abstract are added for the English version (first language) and possible extra native languages (second and third) -- you can comment which ever one is obsolete

\addcontentsline{toc}{chapter}{Abstracts}
%% this is the first abstract of the Master in CyberSecurity master thesis, ALWAYS written in English
\selectlanguage{english}
\masterTitle{Title}{<Insert Title>}
\authorThesis{Author}%in this input frame you write "Author" in the language of the abstract
\Master{Master in Cybersecurity -- <Corporate Strategies / System Design>}% here you have to write the full title of your Master programme
\yearTitle{Academic year}
%% in this environment, you will write your abstract. Keep it limited to a SINGLE PAGE. Max. 500 words should normally do the trick
\begin{abstract}
	\section*{Abstract}\label{sec:abstract}
	\addcontentsline{toc}{section}{\nameref{sec:abstract}}
	%% start adjusting the abstract from here on
	{\fontsize{10}{16}

        The abstrat. A short summary of your motivation, approach, and
findings. Make it interesting so that we keep reading.

	\vspace{10mm}
	\noindent\textbf{Keywords:} MAX. 6 WORDS to summarize/describe the abstract}

% For the students who are working with a company OR have a confidentiality
% clause w.r.t. their Master thesis, uncomment the following lines.
% Set a reasonable disclosure date (e.g., 6 months) up to one year after
% your thesis viva. After that, your thesis will be archived in the
% university library and is publicly accessible (although pretty hard to
% find).
%	\vfill
%	\begin{tcolorbox}[colback=white,colframe=black]
%		\centering
%		\fontsize{10}{12}
%		\textsf{\textbf{Confidential up to and including YYYY-MM-DD\\
%		Do not copy, distribute or make public in any way.}}
%	\end{tcolorbox}
\end{abstract}
\newpage

%%% this is the second abstract of the BruFacE master thesis, written in a second language if desired. This abstract is mainly intended to be written in the native language of the student and the university when they are mutual. This is therefore mainly oriented to Belgian students.
%% For an abstract in another language, not being a native Belgian language, the approval of your supervisor is adviced
%% This example is given in French
\selectlanguage{french}
\masterTitle{Titre}{Ici, vous misez votre titre de votre m\'{e}moire}
\authorThesis{Auteur}%in this input frame you write "Author" in the language of the abstract
\Master{Master en sciences d'ing\'{e}nieur civil \'{e}lectrom\'{e}canicien -- option Technologie des v\'{e}hicules et\\ Transport}% here you have to write the full title of your Master programme
\yearTitle{Ann\'{e}e acad\'{e}mique}
%% in this environment, you will write your abstract. Keep it limited to a SINGLE PAGE. Max. 500 words should normally do the trick
\begin{abstract}
	\section*{R\'{e}sum\'{e}}\label{sec:resume}
	\addcontentsline{toc}{section}{\nameref{sec:resume}}
	{\fontsize{10}{16}
	\lipsum[1-4]
	\vspace{5mm}
	\noindent\textbf{Mots-cl\'{e}s:} MAX. 6 WORDS to summarize/describe the abstract}
%	%% for the students who are working with a company OR have a confidentiality clause w.r.t. their Master thesis, uncomment the following lines
%		\vfill
%		\begin{tcolorbox}[colback=white,colframe=black]
%			\centering
%			\fontsize{10}{12}
%			\textsf{\textbf{Confidentiel jusqu'au dd/mm/20xx\\
%			Ne pas copier, distribuer ou rendre public}}
%		\end{tcolorbox}
\end{abstract}
\newpage
%%% this is the third abstract of the BruFacE master thesis, written in a third language if desired. This abstract is mainly intended to be written in the native language of the student and the university when they are mutual. This is therefore mainly oriented to Belgian students.
%% For an abstract in another language, not being a native Belgian language, the approval of your supervisor is adviced
%% This example is given in Dutch
\selectlanguage{dutch}
\masterTitle{Titel}{Hier zet je de titel van het onderwerp van je Master thesis}
\authorThesis{Auteur}%in this input frame you write "Author" in the language of the abstract
\Master{Master of Science in de ingenieurswetenschappen: werktuigkunde-elektrotechniek -- major in\\Voertuigtechnologie en Transport}% here you have to write the full title of your Master programme
\yearTitle{Academiejaar}
%% in this environment, you will write your abstract. Keep it limited to a SINGLE PAGE. Max. 500 words should normally do the trick
\begin{abstract}
	\section*{Samenvatting}\label{sec:samenvatting}
	\addcontentsline{toc}{section}{\nameref{sec:samenvatting}}
	{\fontsize{10}{16}
	\lipsum[1-4]
	\vspace{5mm}
	\noindent\textbf{Trefwoorden:} MAX. 6 WORDS to summarize/describe the abstract}
%	% for the students who are working with a company OR have a confidentiality clause w.r.t. their Master thesis, uncomment the following lines
%	\vfill
%	\begin{tcolorbox}[colback=white,colframe=black]
%		\centering
%		\fontsize{10}{12}
%		\textsf{\textbf{Vertrouwelijk tot en met dd/mm/20xx\\
%				Niet kopi\"{e}ren, verdelen of publiek bekend maken}}
%	\end{tcolorbox}
\end{abstract}
\newpage

\restoregeometry

%% here comes the preface, if desired -- here you can add personal comments and make acknowledgements (comment if you do not need this inside the thesis)
\setstretch{1.1}
%% here you can write any personal comments and/or acknowledgements towards those who have helped/supported you throughout your thesis. This is recommended, but not necessary
%% when a preface is written, try to limit it to one page as well !!!

\chapter*{Preface}
\addcontentsline{toc}{chapter}{Preface}

\begin{center}
    "I solemnly swear that I'm up to good" (incorrectly inspired by J.K. Rowling)
\end{center}

\vspace{2cm}
Here you can write whatever you desire to make personal comments about your
work.


\clearpage
\chapter*{Acknowledgements}
\addcontentsline{toc}{chapter}{Acknowledgements}

\vspace{2cm}
Here you can thank whoever you see fit to thank including the template authors
:-)


\addtocontents{toc}{\bigskip}% perhaps as well
%%-----------------------------------table of contents--------------------------------------------
\selectlanguage{english}
{\fontsize{12}{16}
	\tableofcontents
	\addcontentsline{toc}{chapter}{Table of Contents}
	%% you can change the structure of letting the lists together on the same page or start a new page each time, that is up to you
	{
	\let\clearpage\relax
	\listoffigures\addcontentsline{toc}{chapter}{List of Figures}
	\listoftables\addcontentsline{toc}{chapter}{List of Tables}
	}%
	{
		\glsaddall % adds all the acronyms and symbols to their respective lists
		\printglossary[type=\acronymtype,title={List of Abbreviations},style=long]
		% list of acronyms
		\let\clearpage\relax
		%\printglossary[type=symbols,style=symbunitlong]   % list of symbols
	}
}%


\newpage
\addtocontents{toc}{\bigskip}% perhaps as well
\chapter{Introduction}\pagenumbering{arabic} %5p
%\section{This is the first section}
\lipsum[1-5] 
\gls{energyconsump} is the power consumption and is defined w.r.t.\ the energy consumption \gls{energyproduction} over a time $\Delta t$ \eqref{eq:eq1}.
\begin{align}
	\gls{energyconsump} = \frac{\gls{energyproduction}}{\Delta t} \equnit{\si[per-mode=symbol]{\joule\per\second}}\label{eq:eq1}
\end{align}
\lipsum[1-15]
\cite{latexcompanion, Menezes2001, Saint-Onge1987, AShamirShareSecret, Rivest:1978:MOD:359340.359342, DBLP:journals/tit/DiffieH76, maslowMotivation, Stanley1874, ElGamal85, RivShaTau01, DBLP:conf/sis/2011}

\section{Motivations}
\subsection{Context}
\subsection{Problem statement}
\section{Project statement \& contributions}% Be clear on what it does provide, does not provide, what is implemented, etc.
\section{Organization of this document}

%\addcontentsline{toc}{part}{State of the Art and requirements}
\addtocontents{toc}{\bigskip}% perhaps as well
\chapter{Literature review, state of the art (SotA), definitions and notations}

\section{Divide and conquer: From title to sub-questions}

Split the main problem into sub-problems, show how related they are and
then explore each with state of the art. What questions need to be answered
leading to what concepts need to be explored in the state of the art?

\section{Notations}

\subsection{Example of potential boxes to make a \LaTeX{} env from}
\begin{bclogo}[arrondi=0.1, logo=\bcquestion, couleur=grey,noborder=true]{Open question}
  example text.
\end{bclogo}


\begin{bclogo}[arrondi=0.1, logo=\bcattention, couleur=grey,noborder=true]{Important remark}
  example text.
\end{bclogo}

\begin{bclogo}[arrondi=0.1, logo=\bcpanchant, couleur=grey,noborder=true]{Restrictions, limitations, work in progress}
  example text.
\end{bclogo}

\begin{bclogo}[arrondi=0.1, logo=\bcinfo, couleur=grey,noborder=true]{Reminder}
  example text.
\end{bclogo}

\begin{bclogo}[arrondi=0.1, logo=\bclampe, couleur=grey,noborder=true]{idea/opportunity/contribution/future work}
  example text.
\end{bclogo}

\begin{bclogo}[arrondi=0.1, logo=\bccrayon, couleur=grey,noborder=true]{Side note, personal thought}
  example text.
\end{bclogo}

\begin{bclogo}[arrondi=0.1, logo=\bccle, couleur=grey,noborder=true]{key element to understand section or to remember from this section}
  example text.
\end{bclogo}

\begin{bclogo}[arrondi=0.1, logo=\bcbombe, couleur=grey,noborder=true]{warning}
  example text.
\end{bclogo}

\begin{bclogo}[arrondi=0.1, logo=\bcbook, couleur=grey,noborder=true]{definition}
  example text.
\end{bclogo}

\section{Definitions}

Where did you find these, based on what criterii, why would that one be the most suitable, etc.

\subsection{List of definitions}

\subsection{Summary of the relationships between the defined concepts}

\section{State of the art and related works}

\subsection{Publications}

\enquote{peer-reviewed} publication (double blinded when possible) are the
core. Other types are \enquote{informational}

\subsubsection{Review methodology}

You are exploring \& comparing others work. What makes your results valid,
relevant, etc.? What are your metrics? How can you ensure you explored all
that needs to be explored? What is you experimental environment and
methodology to measure and compare the tools' performance?

\subsection{Implementations, standards, protocols, technologies, ...}

\subsubsection{Review methodology}

You are exploring \& comparing others work. What makes your results valid, relevant, etc.? What are your metrics? How can you ensure you explored all that needs to be explored? What is you experimental environment and methodology to measure and compare the tools' performance?

\subsection{Summary}

Consider including a table/visual representation to easily grasp the section


%\@endpart

%\addcontentsline{toc}{part}{Contribution}
% Steps of your Plan-Do-Check-Act (PDCA) cycle
\addtocontents{toc}{\bigskip}
% Plan
\chapter{Project's mission, objectives and requirements}
\lipsum[1-2]
% definition of this project's target
% what do we need?
% what does already exists (see previous chapter)?
% what does not yet exist?
% what shall we try to make exist?
% what will remain to do after that?

\section{Requirements}
\lipsum[1-2]
\subsection{List of requirements and how they relate one to another}
\subsection{Requirements covered by state of the art}
\subsection{Requirements not covered by state of the art}

\section{Project scoping}
\lipsum[1-2]
\subsection{Mission statement of this project}
\subsection{Explicit out-of-scope definition}
% what we will not do: because because x, y, z
% Do (and Act, includes the corrective measures)
%10-15p
\chapter{Implementation \& Testing}
% Test driven design
% Design the test first, specify the kpis, the expected behaviours, the use cases, etc. See unit testing methodologies
% Then implement and continously test versus those specifications until implementation is finished
\pagestyle{fancy}
\section{Methodology} % How to conduct the experiment
% How do you test? How do you decide if it was a success? What should you conclude if result is x or y? This needs to be defined before experimenting to be sound and maybe revised after (so conclusion could be that the methodology was flawed)
\subsection{Testing methodology}
\subsection{Design methodology}
\subsection{Implementation methodology}
\section{Setup, Requirements, Environment, Tools \& Materials} % How to reproduce the experiment
\section{Results} %Raw data before analysis

%\section{This is the first section}
\lipsum[1-5] 

\chapter{Experimentation \& data collection}
\pagestyle{fancy}
\section{Methodology} % How to conduct the experiment
% How do you test? How do you decide if it was a success? What should you conclude if result is x or y? This needs to be defined before experimenting to be sound and maybe revised after (so conclusion could be that the methodology was flawed)
\section{Setup, Requirements, Environment, Tools \& Materials} % How to reproduce the experiment
\section{Results} %Raw data before analysis

%\section{This is the first section}
\lipsum[1-5]
% Check: analyse your result, analyse your methodology, revise your thesis, ...
\chapter{Experiment's output/Data Analysis}%15-20p. Process, treatment of collected data and initial conclusion
\pagestyle{fancy}
%\section{This is the first section}
\lipsum[1-5] 

\chapter{CyberSecurity analysis of your project/implementation/solution/proposal}
% your project probably supports cybersecurity one way or another ... the question here is: is your project itself secure? how well is it protected and against what and why?
% Students are often asked that question and some fail to have considered the cybersecurity considerations of their own project. So they design something to control something else but do not secure their own production or did not think about it much or with a structured approach (which should instead be automatic considering the studies)
% if you design a new service e.g. a new ids engine, the first analysis chapter does the analysis of how that new service is useful, efficient e.g. how well your ids engine finds vulnerabilities and the second analysis chapter, this one does the cybersecurity analysis of your new ids (how easy one can attack it etc.)
% Consider the traditionnal risk management workflow from threats, vulnerabilities, risk, impact to mitigations and measurement to check that it works (PDCA)
% Consider showing how your project will be protected, covered by the NIST functions (that you did yourselves or already existing)
% Consider various analysis approaches (offensive approach? how would you attack your system?) including risk management methodologies, audits, architecture review (how well does your project follow existing state of the art guidelines for cybersecurity), ... monarc/ebios/...
\chapter{Discussion}%5-10p. Put things into perspective, take a step back
\section{Comparison with state of the art/related works}%consider including an easy to read table
\section{Lessons learned}% Flaws in methodology, if you/someone else was to do it again, ...
\section{Limitations of validity}% Limitations, pros/cons, discussion on the methodology, usability of the results, ...

\chapter{Future work}
\pagestyle{fancy}
%\section{This is the first section}
\lipsum[1-5] 
%\@endpart

\addtocontents{toc}{\bigskip}% perhaps as well

\chapter{Conclusions}%3-5p
\pagestyle{fancy}
%\section{This is the first section}
\lipsum[1-5] 


% \bibliographystyle{vancouver}
\bibliographystyle{styles/splncs03_2.bst}
\bibliography{references/bibliography}
%\printbibliography
\addcontentsline{toc}{chapter}{Bibliography}

\newpage

\appendix
\addcontentsline{toc}{part}{Appendices}
% code, documentation, tables, raw data, additional things you wrote that might be interesting for the reader, ...
\chapter{Source code}
\cite{ElGamal85, Kerckhoffs, McGarrity2014, Ariely2002, Dossogne2011a, Dossogne2012a, Oakley20140731, Oakley2015, Kruger1999, Dossogne2014}
% use code listing to enhance the readability of your code https://www.overleaf.com/learn/latex/Code_listing
% in the main part of the document, put only what is "essential" for someone to read. In the appendix, you put the rest
\chapter{Documentation}
\section{Digital format documentation}
% It might not be possible for you to "print" all the documentation you made (videos, etc.)
% Links + few word on how it's structured, recommended exploration/reading flow (how it's intended to be watched, which order etc.)
% Screencasting videos (https://github.com/Enselic/recordmydesktop), Asciinema recordings https://asciinema.org/, PowerSession https://github.com/ibigbug/PowerSession, screenshots (PNG + zoom in on the action), ...
\section{Printable documentation}

\chapter{Experimental data}
\chapter{Initial project proposal}
% This section is the document you prepare before reaching/looking for a supervisor. It is, among other things, what helps your supervisor to decide to accept your or not within the research lab and supervise you.
\section{Title}
Title: ...
\section{Proposal summary}
\subsection{Participants}
Project Director/Owner: ...

Researchers(s):
\begin{itemize}
    \item a
    \item b
\end{itemize}

\subsection{Background}
% context, the situation, the problem, ...
\subsection{Objectives}
The goal of this project is to develop ...
% this should do this, that and that
\subsection{Expected outcome}
The project will produce a number of deliverables:
\begin{itemize}
    \item ...
    \item implementation that does x under conditions y ...
    \item a master thesis ...
    \item a draft of a white paper that can be further modified and improved upon for the research lab to consider publishing at a later date (with or without the student's further collaboration, upon his preference)
\end{itemize}

\section{Proposal description (max. 4 pages, ref's included)}
\subsection{Aim of the study and relevance for designated target group}
\subsection{State of the art}
% Describe the state of the art related to this proposal at a national and international level
\subsection{Global research context}
% In which global research context is this proposal situated
\subsection{Research strategy}
% Describe the methods, techniques and procedures by which the research will be conducted
\subsection{Collaboration}
\subsection{Expected outcome}
\subsection{Feasibility \& risks}
% Make your S.W.O.T. analysis. What are your competencies, resources, available time, etc.
\subsection{Yet another section (you can add your own too of course)}
\subsection{References}
% Be careful using the references appropriately

\section{Phasing of the project (max. 4 pages)}

Start date:

End date:

% Here is an example of starting generic workpackages and objectives you might have (you should certainly split it more based on your project)
% # Starting work packages
% Each project has typically a minimum of 5 starting work packages:

% ## Integrate research group
% ### Objective 0: Register/Access communication channels and study its content
% ### Objective 1: Plan/schedule your project to integrate the group (meetings, etc.), its processes and objectives
% ### Objective 2: Check/Act (from PDCA) Reviewing and corrective process
% * Includes presenting your outputs, processes, results and getting it validated (feedback and your modifications)

% ## TechnologyIntelligence
% Start global project shared documentation, technology intelligence (veille technologique, https://en.wikipedia.org/wiki/Technology_intelligence) receptacle & procedures. Outputs data (keywords, authors, venues, ...) as well as processes (code, systems, ...) to monitor and integrate these.
% ### Objective 0: Enumerate & document relevant communication channels and setup/program/code automated tools to monitor them
% * RSS on dblp for example https://dblp.org/faq/How+can+I+fetch+all+publications+of+one+specific+author.html
% * auto import/update from RSS to zotero/mendley group with tagging
% * gits with continuous integration (submodules) and archiving
% * Google Alerts, ...
% ### Objective 1: Explore channels and discover/document/program relevant parameters (keywords, labels, product names, authors, venues, notions, tags, categories, ...).
% Consider generic top tiers cybersecurity venues + venues very specific to your topic
% * Generic CyberSecurity Venues
% 	* https://people.engr.tamu.edu/guofei/sec_conf_stat.htm ![image](uploads/08fc5d627c9b4da024e53a365d4927ae/image.png)
% 	* http://jianying.space/conference-ranking.html ![image](uploads/21d12f24d588107d088c707df3443e27/image.png)
% * Specific to cyber range: International Workshop on Cyber Range Technologies and Applications (CACOE 2020) https://ieeexplore.ieee.org/document/9229838/metrics#metrics
% ### Objective 2: Start your intelligence analysis process (stack the papers, rank them, etc.)
% ### Objective 3: Check/Act (from PDCA) Reviewing and corrective process
% * Includes presenting your outputs, processes, results and getting it validated (feedback and your modifications)

% ## SotA : State of the art
% ### Objective 0: Enumeration of technologies, tools, authors, venues, vendors, ... (input data)
% ### Objective 1: Enumeration & analysis of features, technologies/techniques, pros, cons, restrictions, ... (input data)
% ### Objective 2: Enumeration of project requirements (desired output)
% ### Objective 3: Classification of features & requirements in core/essentials/mandatory, desired, "nice to have", neutral, against (desired output)
% ### Objective 4: Matching/analysis of input versus desired output
% One part will be "this can be reached with that, that and that combined in this way". And another, the reminder, should be "these requirements cannot be satisfied yet with what exist" = your target contribution.
% ### Objective 5: Thesis chapter
% ### Objective 6: Learning process / Documentation chapter
% * For those who are joining the project (they shouldn't need to explore, give them the straight line to what is required to read, learn, experiment)
% ### Objective 7: Check/Act (from PDCA) Reviewing and corrective process
% * Includes presenting your outputs, processes, results and getting it validated (feedback and your modifications)

% ## Roadmap
% Canvassing the project in any foreseeable work packages. Typically, not all will be implemented by the same person. Allows new contributors to join and contribute.
% ### Objective 1: Project's Big Bang: initial burst of creation of packages etc. Enumerate, design, write down, discuss all the work packages, objectives and tasks that you can think of.
% ### Objective 2: Continuous refactoring, updating and integration of new work packages, objectives and tasks.
% ### Objective 3: Check/Act (from PDCA) Reviewing and corrective process
% * Includes presenting your outputs, processes, results and getting it validated (feedback and your modifications)

% ## Prototype
% Create the first prototype using the tools and objectives we have [misc.conventions.objectivestools](misc.conventions.objectivestools)
% ### Objective 0: Establish/Start P.D.C.A. cycle (define objectives, tasks, KPI, ...)
% ### Objective 1: Obtain skills, knowledge and setup labs & other necessary requirements
% ### Objective 2: Gather initial/foreseen requirements
% ### Objective 3: Design testing
% ### Objective 4: Build & test first running prototype.
% Priorities are in this order:
% 1. Make it work (should do what is expected). Deliverable first! If it's "half done" it's not done. If it's easier/faster for someone else with lots of experience to start the project from scratch ... then you have made no contribution.
% 2. Make it usable (it should be easy to use). (same remark as previous)
% 3. Make it nice (the code should be easy to maintain, documented, etc.) (same remark as previous)
% 3. Make it secure
% 4. Make it private
% 5. Make it fast (efficient, does not use many resources)
% ### Objective 5: Program running prototype for the various deployment mode
% see [misc.conventions.objectivestools](misc.conventions.objectivestools)
% ### Objective 6: Continuous evaluation and improvement security
% ### Objective 7: Continuous evaluation and improvement of privacy
% ### Objective 8: Continuous evaluation and improvement with regard to mandatory frameworks (if any)
% ### Objective 9: Continuous integration with the other work packages, projects etc. in the lab
% ### Objective 10: Write corresponding Thesis chapters and documentation
% ### Objective 11: Finish P.D.C.A. cycle (evaluate)

% ## Publication & peer review (documentation, thesis, ...)
% ### Objective 1: Learn, analyse & discuss expectations, rules, evaluation grid
% * Regarding: thesis, documentation, white paper
% * Checkout templates (https://www.overleaf.com/latex/templates/master-in-cybersecurity-be-thesis-template-ulb-unamur-ucl-he2b-slash-esi-helb-erm/ypmhcxmmtgkn), guidelines, evaluation grids, ...

% ### Objective 2: Write and review
% * For each #DocumentTypeToProduce
% 	* For each #Chapter to write do PDCA
% 		* Plan
% 			* Structure the corresponding chapter (train of thoughts, key arguments, liaison, etc. check McGarrity's introduction to public speaking)
% 		* Do
% 			* Fill in the key points in each section (in comments)
% 			* Write properly the first finished version
% 		* Check
% 			* Present results for review (demo, submit for review by whole lab, discuss)
% 		* Act
% 			* Make modifications and document/keep track/document suggestions with examples to share with others
% ### Objective 3: Check & plan publication
% * For each type
% * Check the venue, the conditions
% 	* This includes: what are the requirement to publish your master thesis, deadlines, side, length, layout, format, ... where you need to send to, to whom, etc.

\subsection{Workpackage 1: title}
Start date:

End date:
\subsubsection{Description}
\subsubsection{S.M.A.R.T. Objectives}
% List S.M.A.R.T. objectives https://en.wikipedia.org/wiki/SMART_criteria
\subsubsection{Deliverables and their K.P.I.s}
% Deliverables and how to evaluate them https://en.wikipedia.org/wiki/Performance_indicator
\subsubsection{Participants and responsibility assignment matrix (RACI model)}
% https://en.wikipedia.org/wiki/Responsibility_assignment_matrix
\subsection{Workpackage 2: title}
Start date:

End date:
\subsubsection{Description}
\subsubsection{S.M.A.R.T. Objectives}
% List S.M.A.R.T. objectives https://en.wikipedia.org/wiki/SMART_criteria
\subsubsection{Deliverables and their K.P.I.s}
% Deliverables and how to evaluate them https://en.wikipedia.org/wiki/Performance_indicator
\subsubsection{Participants and responsibility assignment matrix (RACI model)}
% https://en.wikipedia.org/wiki/Responsibility_assignment_matrix
\subsection{Workpackage 3: title}
Start date:

End date:
\subsubsection{Description}
\subsubsection{S.M.A.R.T. Objectives}
% List S.M.A.R.T. objectives https://en.wikipedia.org/wiki/SMART_criteria
\subsubsection{Deliverables and their K.P.I.s}
% Deliverables and how to evaluate them https://en.wikipedia.org/wiki/Performance_indicator
\subsubsection{Participants and responsibility assignment matrix (RACI model)}
% https://en.wikipedia.org/wiki/Responsibility_assignment_matrix
\subsection{Workpackage 4: title}
Start date:

End date:
\subsubsection{Description}
\subsubsection{S.M.A.R.T. Objectives}
% List S.M.A.R.T. objectives https://en.wikipedia.org/wiki/SMART_criteria
\subsubsection{Deliverables and their K.P.I.s}
% Deliverables and how to evaluate them https://en.wikipedia.org/wiki/Performance_indicator
\subsubsection{Participants and responsibility assignment matrix (RACI model)}
% https://en.wikipedia.org/wiki/Responsibility_assignment_matrix


\section{Expertise of the project's research team (max. 2 pages)}
\subsection{Expertise}
% Describe yours and the expertise available to you for this project proposal
\subsection{Publications \& porfolio relevant to the project proposal}
% List of successes, projects, codes, publications relevant to show your ability to succeed in this endeavour

\section{Requirement (equipment, skills, ...)}
% Indicate what is required, how you plan to acquire it, etc. If you need to study an online course first, etc.

\section{Proposal summary table (max. 2 pages)}
% +/- 1 line per field
Domain:

Study director:

Research unit/staff/dept:

Title:

Aim of the study:

Research strategy:

Innovative character:

Target group \& relevance for that audience:
Partnerships:

\section{Evaluation \& Self-Evaluation}
\subsection{Self-Evaluation}
\begin{itemize}
    \item ... /10: Relevance and scope : Is the relevance of the proposed research well defined? Is the scope of project well described and delimited?
    \item ... /10: Efficiency : Are the required/allocated means means, i.e. personnel, infrastructure and equipment, consistent with the projected outcome of the project?
    \item ... /10: State of the Art : How is the state of the art, related to this proposal at a national and international level, described? Are the proposers aware of past and current similar research activities?
    \item ... /10: Research Strategy and Phasing : Are the phases of the project realistic, coherent and in line with the intended objectives? How realistic and well argumented are the objectives? Is the research team well organized and able to perform the planned research
    \item Remarks: 
\end{itemize}

\subsection{Evaluation}
\begin{itemize}
    \item ... /10: Relevance and scope
    \item ... /10: Efficiency
    \item ... /10: State of the Art
    \item ... /10: Research Strategy and Phasing
    \item Remarks: 
\end{itemize}
\chapter{Project management records / artifacts}
\section{Gantt charts}
\section{Pomodoro charts}
\section{Agile Methodology's User stories}
\section{Workpackages, Objectives and tasks}
\section{Diagrams \& user produced artifacts: use cases, user stories, UML, ...}
\end{document}          