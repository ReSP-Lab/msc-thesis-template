\chapter{Literature review, state of the art (SotA), definitions and notations}

\section{Divide and conquer: From title to sub-questions}

Split the main problem into sub-problems, show how related they are and
then explore each with state of the art. What questions need to be answered
leading to what concepts need to be explored in the state of the art?

\section{Notations}

\subsection{Example of potential boxes to make a \LaTeX{} env from}
\begin{bclogo}[arrondi=0.1, logo=\bcquestion, couleur=grey,noborder=true]{Open question}
  example text.
\end{bclogo}


\begin{bclogo}[arrondi=0.1, logo=\bcattention, couleur=grey,noborder=true]{Important remark}
  example text.
\end{bclogo}

\begin{bclogo}[arrondi=0.1, logo=\bcpanchant, couleur=grey,noborder=true]{Restrictions, limitations, work in progress}
  example text.
\end{bclogo}

\begin{bclogo}[arrondi=0.1, logo=\bcinfo, couleur=grey,noborder=true]{Reminder}
  example text.
\end{bclogo}

\begin{bclogo}[arrondi=0.1, logo=\bclampe, couleur=grey,noborder=true]{idea/opportunity/contribution/future work}
  example text.
\end{bclogo}

\begin{bclogo}[arrondi=0.1, logo=\bccrayon, couleur=grey,noborder=true]{Side note, personal thought}
  example text.
\end{bclogo}

\begin{bclogo}[arrondi=0.1, logo=\bccle, couleur=grey,noborder=true]{key element to understand section or to remember from this section}
  example text.
\end{bclogo}

\begin{bclogo}[arrondi=0.1, logo=\bcbombe, couleur=grey,noborder=true]{warning}
  example text.
\end{bclogo}

\begin{bclogo}[arrondi=0.1, logo=\bcbook, couleur=grey,noborder=true]{definition}
  example text.
\end{bclogo}

\section{Definitions}

Where did you find these, based on what criterii, why would that one be the most suitable, etc.

\subsection{List of definitions}

\subsection{Summary of the relationships between the defined concepts}

\section{State of the art and related works}

\subsection{Publications}

\enquote{peer-reviewed} publication (double blinded when possible) are the
core. Other types are \enquote{informational}

\subsubsection{Review methodology}

You are exploring \& comparing others work. What makes your results valid,
relevant, etc.? What are your metrics? How can you ensure you explored all
that needs to be explored? What is you experimental environment and
methodology to measure and compare the tools' performance?

\subsection{Implementations, standards, protocols, technologies, ...}

\subsubsection{Review methodology}

You are exploring \& comparing others work. What makes your results valid, relevant, etc.? What are your metrics? How can you ensure you explored all that needs to be explored? What is you experimental environment and methodology to measure and compare the tools' performance?

\subsection{Summary}

Consider including a table/visual representation to easily grasp the section

