\chapter{Experiment's output/Data Analysis}%15-20p. Process, treatment of collected data and initial conclusion
\pagestyle{fancy}
%\section{This is the first section}
\lipsum[1-5] 

\chapter{CyberSecurity analysis of your project/implementation/solution/proposal}
% your project probably supports cybersecurity one way or another ... the question here is: is your project itself secure? how well is it protected and against what and why?
% Students are often asked that question and some fail to have considered the cybersecurity considerations of their own project. So they design something to control something else but do not secure their own production or did not think about it much or with a structured approach (which should instead be automatic considering the studies)
% if you design a new service e.g. a new ids engine, the first analysis chapter does the analysis of how that new service is useful, efficient e.g. how well your ids engine finds vulnerabilities and the second analysis chapter, this one does the cybersecurity analysis of your new ids (how easy one can attack it etc.)
% Consider the traditionnal risk management workflow from threats, vulnerabilities, risk, impact to mitigations and measurement to check that it works (PDCA)
% Consider showing how your project will be protected, covered by the NIST functions (that you did yourselves or already existing)
% Consider various analysis approaches (offensive approach? how would you attack your system?) including risk management methodologies, audits, architecture review (how well does your project follow existing state of the art guidelines for cybersecurity), ... monarc/ebios/...